\documentclass[10pt,a4paper]{article}
\usepackage[latin1]{inputenc}
\usepackage{amsmath}
\usepackage{amsfonts}
\usepackage{amssymb}
\usepackage{makeidx}
\usepackage{graphicx}
\usepackage[ruled,vlined]{algorithm2e}
\author{Artem Los}
\title{Hello}

\usepackage{marginnote}
\usepackage{verbatim} % for the box
\usepackage{fancyvrb} % for the box

\usepackage{listings}
\usepackage{hyperref}

\lstset{
% vilket spr�k vi anv�nder i v�ra kodlistings, s� att listings-paketet vet hur den ska highligta saker
language=Java,
% huruvida vi ska ha syntax highlighting
fancyvrb=true,
% hur stora tabstopp vi ska ha
tabsize=4,
% huruvida vi ska till�ta andra tecken �n a-z
extendedchars=\true
% hur breda listings vi vill ha (skriv exempelvis linewidth=0.5\textwidth f�r att f� listings som bara tar upp halva bredden av sidan)
linewidth=\textwidth,
% huruvida vi ska visa mellanslag
showstringspaces=false,
% huruvida vi ska bryta rader som �r f�r l�nga
breaklines=true,
% huruvida den ska f� bryta rader mitt i ord eller inte (true h�r betyder att den bara bryter mellan ord)
breakatwhitespace=true,
% indentera radbrytningar automatiskt
breakautoindent=true,
% l�gg in radnummer p� v�nster sida
numbers=left,
% hur stora radnumren ska vara
numberstyle=\tiny,
% hur l�ngt det ska vara mellan radnumren och koden
numbersep=8pt
}

\usepackage{pgf}
\usepackage{pgfpages}

\usepackage{fullpage}  % might require you to compile the page several times.



\begin{document}
\section*{Matching}
\paragraph{\color{blue} Removing the go statement} from \verb|Seek| will turn the program into a synchronous one, where all steps can be predicted. Anna is going to send a message to Bob, Cody to Dave, and Eva to no one.
\paragraph{\color{blue} Changing wait group} declarations will throw an error because we don't pass the wait group by reference to \verb|Seek| method and so the wait group inside \verb|Seek| isn't able to communicate with the one in \verb|main|. Two messages will be sent/received and the last one will fail as all go routines are asleep.

\paragraph{\color{blue} Removing the buffer} will thrown an error because unbuffered channels accept sends only if there is a corresponding receive. There doesn't seem to be a clear receiver of the sent information through the channel.
\begin{quotation}
	By default channels are unbuffered, meaning that they will only accept sends (chan <-) if there is a corresponding receive (<- chan) ready to receive the sent value. Buffered channels accept a limited number of values without a corresponding receiver for those values.\footnote{https://gobyexample.com/channel-buffering, accessed 2015.04.14}
\end{quotation}

\paragraph{\color{blue} Removing default} will not cause any problems in the current set up as it is constructed in such a way that one will be left unmatched. However, decreasing the size of the array of people, for example by removing \textit{Eva} will throw a an error (a deadlock) because there is no one in the match channel left and it is still open. Thus, we are waiting for something that never arrives, which leads to a deadlock. A solution would be to add \verb|close(match)| right after \verb|wg.Wait()|.

\section*{Julia}
The program uses all CPUs. The speed improved in the order of 20s.
It seems like some might argue that we should have go routines for each pixel, but I think that we should have four that can be spread out on 4 processors.
\lstinputlisting{njulia.go}

\section*{Weather Station}
\lstinputlisting{njulia.go}

\end{document}

