\documentclass[10pt,a4paper]{article}
\usepackage[latin1]{inputenc}
\usepackage{amsmath}
\usepackage{amsfonts}
\usepackage{amssymb}
\usepackage{makeidx}
\usepackage{graphicx}
\usepackage[ruled,vlined]{algorithm2e}
\author{Artem Los}
\title{Hello}

\usepackage{marginnote}
\usepackage{verbatim} % for the box
\usepackage{fancyvrb} % for the box

\usepackage{listings}

\lstset{
% vilket spr�k vi anv�nder i v�ra kodlistings, s� att listings-paketet vet hur den ska highligta saker
language=Java,
% huruvida vi ska ha syntax highlighting
fancyvrb=true,
% hur stora tabstopp vi ska ha
tabsize=4,
% huruvida vi ska till�ta andra tecken �n a-z
extendedchars=\true
% hur breda listings vi vill ha (skriv exempelvis linewidth=0.5\textwidth f�r att f� listings som bara tar upp halva bredden av sidan)
linewidth=\textwidth,
% huruvida vi ska visa mellanslag
showstringspaces=false,
% huruvida vi ska bryta rader som �r f�r l�nga
breaklines=true,
% huruvida den ska f� bryta rader mitt i ord eller inte (true h�r betyder att den bara bryter mellan ord)
breakatwhitespace=true,
% indentera radbrytningar automatiskt
breakautoindent=true,
% l�gg in radnummer p� v�nster sida
numbers=left,
% hur stora radnumren ska vara
numberstyle=\tiny,
% hur l�ngt det ska vara mellan radnumren och koden
numbersep=8pt
}

\usepackage{pgf}
\usepackage{pgfpages}

\usepackage{fullpage}  % might require you to compile the page several times.



\begin{document}
\section*{Exercise HashMap}

\begin{tabular}{lllll}
	\hline
	\textbf{Unsorted Vector} &\textbf{Sorted Vector} & \textbf{Unsorted LinkedList} & \textbf{Sorted LinkedList} & \textbf{Hash table}  \\
	O(1) & O(1) & O(n) & O(n) & O(1)\\
	O(n) & O(n) & O(1) & O(1) & O(1) \\
	O(n) & O(n) & O(1) & O(1) & O(1)\\
	\hline
\end{tabular}
It is assumed that by either \textit{adding} or \textit{removing} an element in a LinkedList, we refer to the last element. Otherwise, the time complexity would be $O(n)$.

\subsection*{Comments}
\textbf{Vectors} are great because each \textit{look up} is performed in constant time, while \textit{adding} and \textit{removing} takes linear time. The latter is because we when we add a new item, we actually create a new vector and thus we have to copy all the values from the old vector to the new one.
\textbf{Linked Lists} are good at \textit{adding} and \textit{removing} items that are either in the beginning or the end of a list. However, the \textit{search} operation is performed in linear time as we need to go through the entire list (in general).
\textbf{Hash table} aims to perform \textit{search}, \textit{adding} and \textit{removing} in constant time. Occasionally, we will have to resize the table, which will be costly, but on average, we still get constant time for each operation.

\subsection*{StringHash class}
\lstinputlisting{files/StringHash.java}

\subsection*{StringHashTest class}
\lstinputlisting{files/StringHashTest.java}



\end{document}

