\documentclass[10pt,a4paper]{article}
\usepackage[latin1]{inputenc}
\usepackage{amsmath}
\usepackage{amsfonts}
\usepackage{amssymb}
\usepackage{makeidx}
\usepackage{graphicx}
\author{Artem Los}
\title{Hello}

\usepackage{marginnote}
\usepackage{verbatim} % for the box
\usepackage{fancyvrb} % for the box

\usepackage{listings}

\lstset{language=Java,
numberstyle=\footnotesize,
basicstyle=\ttfamily\footnotesize,
numbers=left,
rulesepcolor=\color{gray}, %fancy shadow for borders
stepnumber=1,
frame=shadowbox,
breaklines=true}

\usepackage{pgf}
\usepackage{pgfpages}

\usepackage{fullpage}  % might require you to compile the page several times.



\begin{document}

\section*{Exercises 4.40-4.42 and 4.54-4.55}
\textbf{The Club class} is shown below:
\begin{lstlisting}
/**
 * Store details of club memberships.
 * 
 * @author (Artem Los) 
 * @version (2014.09.29)
 */
import java.util.ArrayList;
import java.util.Iterator;

public class Club
{
    private ArrayList<Membership> memberships;

    public Club()
    {
        memberships = new ArrayList<Membership>();
        
    }

    /**
     * Add a new member to the club's list of members.
     * @param member The member object to be added.
     */
    public void join(Membership member)
    {
        memberships.add(member);
    }

    /**
     * @return The number of members (Membership objects) in
     *         the club.
     */
    public int numberOfMembers()
    {
        return memberships.size(); 
    }
    
    /**
     * Determine the number of members who joinded in the given month. 
     * @param month The month we are interested in.
     * @return The number of members who joined in that month.
     */
    public int joinedInMonth(int month) throws IllegalArgumentException
    {
        int count = 0;
        
        if(month < 1 || month > 12) {
            throw new IllegalArgumentException(
                "Month " + month + " out of range. Must be in the range 1 ... 12");
        }
        
        for(Membership member : memberships )
        {
            if(member.getMonth() == month)
            {
                count++;
            }
        }
        
        return count;
    }
    
    /**
     * Remove from the club's collection all members who
     * joined in the given month, and return them stored
     * in a separate collection object.
     * @param month The month of the membership.
     * @param year The year of the membership.
     * @return The members who joined in the given month and year
     */
    public ArrayList<Membership> purge(int month, int year)
    {
        if(month < 1 || month > 12) {
            System.out.println("Month " + month + " out of range. Must be in the range 1 ... 12");
            return new ArrayList<Membership>();
        }
        
        Iterator<Membership> it = memberships.iterator();
        
        ArrayList<Membership> toReturn = new ArrayList<Membership>();
            
        while(it.hasNext())
        {
            Membership m = it.next();
            
            if(m.getMonth() == month && m.getYear()==year)
            {
                toReturn.add(m);
                it.remove();
            }
        }
        
        return toReturn;
    }
}
\end{lstlisting}

\section*{Exercises 4.48-4.52}
\textit{Answer for Exercise 4.50}: The \textit{getLot} method is entirely based on the array's size, thus removing an item will decrease the size of the array in such a way that the \textit{lotNumber} will no longer correspond to the actual \textit{lot} number.

\verb|Lot selectedLot = lots.get(lotNumber - 1);|

If we instead create a for loop that will check individual lots and compare to the \textit{lotNumber} requested, the size of the array (which works kind of like an internal counter) will no longer matter.
\\
\\
\textbf{The Action class} is shown below:
\begin{lstlisting}
import java.util.ArrayList;
import java.util.Iterator;

public class Auction
{
    // The list of Lots in this auction.
    private ArrayList<Lot> lots;
    // The number that will be given to the next lot entered
    // into this auction.
    private int nextLotNumber;

    /**
     * Create a new auction.
     */
    public Auction()
    {
        lots = new ArrayList<Lot>();
        nextLotNumber = 1;
    }


    public void enterLot(String description)
    {
        lots.add(new Lot(nextLotNumber, description));
        nextLotNumber++;
    }


    public void showLots()
    {
        for(Lot lot : lots) {
            System.out.println(lot.toString());
        }
    }
    

    public void makeABid(int lotNumber, Person bidder, long value)
    {
        Lot selectedLot = getLot(lotNumber);
        if(selectedLot != null) {
            Bid bid = new Bid(bidder, value);
            boolean successful = selectedLot.bidFor(bid);
            if(successful) {
                System.out.println("The bid for lot number " +
                                   lotNumber + " was successful.");
            }
            else {
                // Report which bid is higher.
                Bid highestBid = selectedLot.getHighestBid();
                System.out.println("Lot number: " + lotNumber +
                                   " already has a bid of: " +
                                   highestBid.getValue());
            }
        }
    }


    public Lot getLot(int lotNumber)
    {
        if((lotNumber >= 1) && (lotNumber < nextLotNumber)) {
            
            
            for(Lot lot : lots)
            {
                if(lot.getNumber() == lotNumber)
                {
                    return lot;
                }
            }
            
            return null;
            
        }
        else {
            System.out.println("Lot number: " + lotNumber +
                               " does not exist.");
            return null;
        }
    }
    
    public void close()
    {
        for(Lot lot : lots)
        {
            Bid highestBid = lot.getHighestBid();
            
            if(highestBid != null)
            {
                System.out.println(lot.toString() + " by " + highestBid.getBidder().getName());
            }
            else
            {
                System.out.println(lot.toString() + " does not have any bidders");
            }
        }
    }
    
    public ArrayList<Lot> getUnsold()
    {
        ArrayList<Lot> unsoldLots = new ArrayList<Lot>();
        
        int count = 0;
        
        for(Lot lot : this.lots)
        {
            if(lot.getHighestBid() == null)
            {
                unsoldLots.add(this.lots.get(count));
            }
            count++;
        }
        
        return unsoldLots;
    }
    
    /**
     * Remove the lot with the given lot number
     * @param number The number of the lot to be removed.
     * @return The lot with the given number, or null if there is no such lot.
     */
    public Lot removeLot(int number)
    {       
        if((number >= 1) && (number < nextLotNumber)) {
            
            Iterator<Lot> it = lots.iterator();
            
            while(it.hasNext())
            {
                Lot lot = it.next();
                
                if(lot.getNumber() == number)
                {
                    lots.remove(lot);
                    return lot;
                }
            }
            
            return null;
            
        }
        else {
            System.out.println("Lot number: " + number +
                               " does not exist.");
            return null;
        }
    }
    
}
\end{lstlisting}
\newpage
\section*{Exercises 4.56-4.59}
The \textbf{StockManager} class is illustrated below:
\begin{lstlisting}
import java.util.ArrayList;

/**
 * Manage the stock in a business.
 * The stock is described by zero or more Products.
 * 
 * @author (Artem Los) 
 * @version (2014.09.29)
 */
public class StockManager
{
    // A list of the products.
    private ArrayList<Product> stock;

    /**
     * Initialise the stock manager.
     */
    public StockManager()
    {
        stock = new ArrayList<Product>();
    }

    /**
     * Add a product to the list.
     * @param item The item to be added.
     */
    public void addProduct(Product item)
    {
        stock.add(item);
    }
    
    /**
     * Receive a delivery of a particular product.
     * Increase the quantity of the product by the given amount.
     * @param id The ID of the product.
     * @param amount The amount to increase the quantity by.
     */
    public void delivery(int id, int amount)
    {
        Product product = findProduct(id);
        
        if(product != null)
        {
            product.increaseQuantity(amount);
        }
    }
    
    /**
     * Try to find a product in the stock with the given id.
     * @return The identified product, or null if there is none
     *         with a matching ID.
     */
    public Product findProduct(int id)
    {
        for(Product product : stock)
        {
            if(product.getID() == id)
            {
                return product;
            }
        }
        return null;
    }
    
    /**
     * Locate a product with the given ID, and return how
     * many of this item are in stock. If the ID does not
     * match any product, return zero.
     * @param id The ID of the product.
     * @return The quantity of the given product in stock.
     */
    public int numberInStock(int id)
    {
        Product product = findProduct(id);
        
        if(product != null)
        {
            return product.getQuantity();
        }
        
        return 0;
    }

    /**
     * Print details of all the products.
     */
    public void printProductDetails()
    {
        for(Product product : stock)
        {
            System.out.println(product.toString());
        }
        
    }
}
\end{lstlisting}
\end{document}

