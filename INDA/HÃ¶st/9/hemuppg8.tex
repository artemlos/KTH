\documentclass[10pt,a4paper]{article}
\usepackage[latin1]{inputenc}
\usepackage{amsmath}
\usepackage{amsfonts}
\usepackage{amssymb}
\usepackage{makeidx}
\usepackage{graphicx}
\usepackage[ruled,vlined]{algorithm2e}
\author{Artem Los}
\title{Hello}

\usepackage{marginnote}
\usepackage{verbatim} % for the box
\usepackage{fancyvrb} % for the box

\usepackage{listings}

\lstset{
% vilket spr�k vi anv�nder i v�ra kodlistings, s� att listings-paketet vet hur den ska highligta saker
language=Java,
% huruvida vi ska ha syntax highlighting
fancyvrb=true,
% hur stora tabstopp vi ska ha
tabsize=4,
% huruvida vi ska till�ta andra tecken �n a-z
extendedchars=\true
% hur breda listings vi vill ha (skriv exempelvis linewidth=0.5\textwidth f�r att f� listings som bara tar upp halva bredden av sidan)
linewidth=\textwidth,
% huruvida vi ska visa mellanslag
showstringspaces=false,
% huruvida vi ska bryta rader som �r f�r l�nga
breaklines=true,
% huruvida den ska f� bryta rader mitt i ord eller inte (true h�r betyder att den bara bryter mellan ord)
breakatwhitespace=true,
% indentera radbrytningar automatiskt
breakautoindent=true,
% l�gg in radnummer p� v�nster sida
numbers=left,
% hur stora radnumren ska vara
numberstyle=\tiny,
% hur l�ngt det ska vara mellan radnumren och koden
numbersep=8pt
}

\usepackage{pgf}
\usepackage{pgfpages}

\usepackage{fullpage}  % might require you to compile the page several times.



\begin{document}
\section*{Exercise 'NIC programs'}


\subsection*{Multiplication of an array}
\lstinputlisting{mulArray.as}
\subsection*{Moving around things in memory}
\lstinputlisting{movestuff.as}
\subsection*{Fibonacci generator}
\lstinputlisting{fibonacci.as}

\section*{Exercise 'wost case ordo and ordo in general'}
\subsection*{Loop 1}
I chose the for loop as the \textit{loop invariance}.
Then, $O(n)$.
\subsection*{Loop 2}
The $i$ is never increased. Thus the array will continue forever. There is no worst case then. NB: If this is a typo, and it should say $i++$ in the array, then $O(n)$.

\subsection*{Loop 3}
The $i$ is never increased, hence no worst case. If it's assumed that $i$ is increased by $1$, we have $O(n^2)$.

\subsection*{Loop 4}
I assume that all for-loops have an implicit increment of $1$ on each iteration for the variable given in the initialization. Then, $O(n^2)$

\subsection*{Loop 5}
The same assumption regarding the dummy variable that keeps track of the index. Then, $O(n^4)$. We will sum $1+2+3+4+5 + \dots  + x$ $n^2$ times.

\subsection*{Exercise 'explain ordo'}
\textbf{Statement} We want to show that $(n+1)^3 = O(n^3)$.\\
\textbf{Explanation} Let's expand LHS.
\begin{eqnarray*}
(n+1)^3 = n^3 + 3n^2 + 3n + 1
\end{eqnarray*}
The definition states that $\exists c > 0, \exists n_0 > 0 : f(n) \le cg(n), \forall n\ge n_0$. In our case, we want to find positive constants $c,n_0$ such that $n^3+3n^2 + 3n +1 \le cn^3 $ for all $n\ge n_0$. 

Let's divide both sides by $n^3$, i.e.

\begin{eqnarray*}
\frac{n^3+3n^2+3n+1}{n^3} \le c \\
1 + \frac{3}{n} + \frac{3}{n^2} + \frac{1}{n^3} \le c\\
\end{eqnarray*}

Already at this stage we see that the constant will be finite given our restriction that $n> n_0$, where $n_0 > 0$. In fact, as $n \to \infty$, $c$ can be as little as $1$.$\square$

\subsection*{Exercise 'reverse algorithm analysis'}
The algorithm will have $T(n) = O(n)$. The key operation is the number of times the for loop is being executed. That's because the operation involved inside the for loop have a constant time, i.e. $O(1)$, so it won't affect the final ordo expression. It's $O(n)$, even if the algorithm will take $T(n)=n/2$ times, assuming we count the number of times the loop is being executed.

If the array contains identical elements, then the swap operation is not going to be executed. Assuming it takes a constant amount of time to execute it, no noticeable change is going to be observed. Maybe, some more time, depending on the how much time the \textit{swap} operation takes. 


\end{document}

