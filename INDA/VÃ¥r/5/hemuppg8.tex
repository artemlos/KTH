\documentclass[10pt,a4paper]{article}
\usepackage[latin1]{inputenc}
\usepackage{amsmath}
\usepackage{amsfonts}
\usepackage{amssymb}
\usepackage{makeidx}
\usepackage{graphicx}
\usepackage[ruled,vlined]{algorithm2e}
\author{Artem Los}
\title{Hello}

\usepackage{marginnote}
\usepackage{verbatim} % for the box
\usepackage{fancyvrb} % for the box

\usepackage{listings}

\lstset{
% vilket spr�k vi anv�nder i v�ra kodlistings, s� att listings-paketet vet hur den ska highligta saker
language=Java,
% huruvida vi ska ha syntax highlighting
fancyvrb=true,
% hur stora tabstopp vi ska ha
tabsize=4,
% huruvida vi ska till�ta andra tecken �n a-z
extendedchars=\true
% hur breda listings vi vill ha (skriv exempelvis linewidth=0.5\textwidth f�r att f� listings som bara tar upp halva bredden av sidan)
linewidth=\textwidth,
% huruvida vi ska visa mellanslag
showstringspaces=false,
% huruvida vi ska bryta rader som �r f�r l�nga
breaklines=true,
% huruvida den ska f� bryta rader mitt i ord eller inte (true h�r betyder att den bara bryter mellan ord)
breakatwhitespace=true,
% indentera radbrytningar automatiskt
breakautoindent=true,
% l�gg in radnummer p� v�nster sida
numbers=left,
% hur stora radnumren ska vara
numberstyle=\tiny,
% hur l�ngt det ska vara mellan radnumren och koden
numbersep=8pt
}

\usepackage{pgf}
\usepackage{pgfpages}

\usepackage{fullpage}  % might require you to compile the page several times.



\begin{document}
\section*{Exercise BinarySearchTree}
\subsection*{BinarySearchTree time complexity}
\begin{tabular}{lllll}
	\hline
	\textbf{Operation} & \textbf{Time complexity} \\
	Find & O(n) \\
	Insert & O(n) \\
	NumberOfElements & O(1) \\
	Depth & O(n) \\
	NumberOfLeaves & O(n) \\
	ToString & O(n) \\
	\hline
\end{tabular}

\subsection*{Treap time complexity}
\begin{tabular}{lllll}
	\hline
	\textbf{Operation} & \textbf{Time complexity} \\
	Find & O(log n) \\
	Insert & O(log n) \\
	NumberOfElements & O(1) \\
	Depth & O(n) \\
	NumberOfLeaves & O(n) \\
	ToString & O(n) \\
	\hline
\end{tabular}

\subsection*{Tree class}
\lstinputlisting{Tree.java}

\subsection*{BinarySearchTreeTest class}
\lstinputlisting{BinarySearchTreeTest.java}



\end{document}

