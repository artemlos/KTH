\documentclass[10pt,a4paper]{article}
\usepackage[latin1]{inputenc}
\usepackage{amsmath}
\usepackage{amsfonts}
\usepackage{amssymb}
\usepackage{makeidx}
\usepackage{graphicx}
\usepackage[ruled,vlined]{algorithm2e}
\author{Artem Los}
\title{Hello}

\usepackage{marginnote}
\usepackage{verbatim} % for the box
\usepackage{fancyvrb} % for the box

\usepackage{listings}

\lstset{
% vilket spr�k vi anv�nder i v�ra kodlistings, s� att listings-paketet vet hur den ska highligta saker
language=Java,
% huruvida vi ska ha syntax highlighting
fancyvrb=true,
% hur stora tabstopp vi ska ha
tabsize=4,
% huruvida vi ska till�ta andra tecken �n a-z
extendedchars=\true
% hur breda listings vi vill ha (skriv exempelvis linewidth=0.5\textwidth f�r att f� listings som bara tar upp halva bredden av sidan)
linewidth=\textwidth,
% huruvida vi ska visa mellanslag
showstringspaces=false,
% huruvida vi ska bryta rader som �r f�r l�nga
breaklines=true,
% huruvida den ska f� bryta rader mitt i ord eller inte (true h�r betyder att den bara bryter mellan ord)
breakatwhitespace=true,
% indentera radbrytningar automatiskt
breakautoindent=true,
% l�gg in radnummer p� v�nster sida
numbers=left,
% hur stora radnumren ska vara
numberstyle=\tiny,
% hur l�ngt det ska vara mellan radnumren och koden
numbersep=8pt
}

\usepackage{pgf}
\usepackage{pgfpages}

\usepackage{fullpage}  % might require you to compile the page several times.



\begin{document}
\section*{Exercise 9.11}
The \textit{Device} class must have a definition of \textit{getName} method, because \textit{Device} class is the static type.
\section*{Exercise 9.12}
At runtime, the \textit{getName} that is defined in the dynamic type - \textit{Printer} class - will be executed.
\section*{Exercise 9.13}
All classes inherit from \textit{Object} class, so if the \textit{Student} class does not override the \verb|toString| method, the one in \verb|object.toString| will be used. This will print out the class name and a memory address. These lines will compile.
\section*{9.14}
The following lines will compile. \verb|System.out.println()| will search for \verb|toString| method when attempting to get a representation of the object. This will return the same piece of information in \textit{Exercise 9.13}, eg. \verb|Student@43b6c732|.
\section*{Exercise 9.15}
Since the \textit{Object} class has a method \verb|toString|, this code will compile. At runtime, the compiler will check if this method exists in the dynamic type, so if we have overridden it with another method in the \textit{Student} class, the custom method will be executed, i.e. the one in the \textit{Student} class.
\section*{Exercise 9.16}
\begin{lstlisting}
T x = new D();
\end{lstlisting}

\section*{Linked List}
\subsection*{Time complexity}
\textbf{isHealthy} = $O(n)$\\
\textbf{LinkedList} = $O(1)$\\
\textbf{addFirst} = $O(1)$\\
\textbf{addLast} = $O(1)$\\
\textbf{getFirst} = $O(1)$\\
\textbf{getLast} = $O(1)$\\
\textbf{get} = $O(n)$\\
\textbf{removeFirst} = $O(1)$\\
\textbf{clear} = $O(1)$\\
\textbf{size} = $O(1)$\\
\textbf{isEmpty} = $O(1)$\\
\textbf{toString} = $O(n)$\\

\newpage
\subsection*{Source code}
\subsubsection*{LinkedList class}
\lstinputlisting{LinkedList.java}
\subsubsection*{LinkedListTest class}
\lstinputlisting{LinkedListTest.java}


\end{document}

