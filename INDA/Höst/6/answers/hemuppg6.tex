\documentclass[10pt,a4paper]{article}
\usepackage[latin1]{inputenc}
\usepackage{amsmath}
\usepackage{amsfonts}
\usepackage{amssymb}
\usepackage{makeidx}
\usepackage{graphicx}
\author{Artem Los}
\title{Hello}

\usepackage{marginnote}
\usepackage{verbatim} % for the box
\usepackage{fancyvrb} % for the box

\usepackage{listings}

\lstset{
% vilket spr�k vi anv�nder i v�ra kodlistings, s� att listings-paketet vet hur den ska highligta saker
language=Java,
% huruvida vi ska ha syntax highlighting
fancyvrb=true,
% hur stora tabstopp vi ska ha
tabsize=4,
% huruvida vi ska till�ta andra tecken �n a-z
extendedchars=\true
% hur breda listings vi vill ha (skriv exempelvis linewidth=0.5\textwidth f�r att f� listings som bara tar upp halva bredden av sidan)
linewidth=\textwidth,
% huruvida vi ska visa mellanslag
showstringspaces=false,
% huruvida vi ska bryta rader som �r f�r l�nga
breaklines=true,
% huruvida den ska f� bryta rader mitt i ord eller inte (true h�r betyder att den bara bryter mellan ord)
breakatwhitespace=true,
% indentera radbrytningar automatiskt
breakautoindent=true,
% l�gg in radnummer p� v�nster sida
numbers=left,
% hur stora radnumren ska vara
numberstyle=\tiny,
% hur l�ngt det ska vara mellan radnumren och koden
numbersep=8pt
}

\usepackage{pgf}
\usepackage{pgfpages}

\usepackage{fullpage}  % might require you to compile the page several times.



\begin{document}
\section*{Exercises 4.73-4.78}
The \textbf{LogAnalyzer} class is shown below:
\begin{lstlisting}
public class LogAnalyzer
{
    // Where to calculate the hourly access counts.
    private int[] hourCounts;
    // Use a LogfileReader to access the data.
    private LogfileReader reader;

    /**
     * Create an object to analyze hourly web accesses.
     */
    public LogAnalyzer()
    { 
        // Create the array object to hold the hourly
        // access counts.
        hourCounts = new int[24];
        // Create the reader to obtain the data.
        reader = new LogfileReader();
    }

    /**
     * Analyze the hourly access data from the log file.
     */
    public void analyzeHourlyData()
    {
        while(reader.hasNext()) {
            LogEntry entry = reader.next();
            int hour = entry.getHour();
            hourCounts[hour]++;
        }
    }
    /**
     * Return the number of  accesses recorded in the log file.
     */
    public int numberOfAccesses()
    {
    	int total = 0;
    	
    	/*
    	 * Add the value in each element of hourCounts to
    	 * to total
    	 */
    	
    	for (int hours : hourCounts) {
			total += hours;
		}
    	
    	return total;
    }
    /**
     * Returns the busiest hour.
     */
    public int busiestHour()
    {
    	// we assume that the smallest element is zero.
    	// if this is not the case, use Integer.MIN_VALUE.
    	int busiestHour = 0;
    	
    	for(int hour : hourCounts)
    	{
    		if(hour > busiestHour)
    			busiestHour = hour;
    	}
    	
    	return busiestHour;
    }
    
    /**
     * Returns the quietestHour.
     */
    public int quietestHour()
    {
    	// we could possibly set this to busiestHour()
    	// but that would be to much work.
    	// also, for general case, use
    	// hourCounts.length > 0 ? hourCounts[0] : 0
    	int quietestHour =  hourCounts[0];
    	
    	for (int hour : hourCounts)
    	{
			if(hour < quietestHour)
				quietestHour = hour;
		}
    	
    	return quietestHour;
    }
    /**
     * Returns the busiest two hour period.
     * @return The first hour of that interval.
     */
    public int busiestTwoHour()
    {
    	int busiestTwoHour = 0;
        int busiestHour = 0;
    	
        // the 23 could be replaced with
        // hourCounts.length - hourCounts.length % 2 -1
        // if the array size would not be known in advance.
        
    	for(int i = 0; i < 23; i++)
    	{
    		if(hourCounts[i] + hourCounts[i+1] > busiestTwoHour)
    		{
    			busiestHour = hourCounts[i];
    			busiestTwoHour = hourCounts[i]+ hourCounts[i+1];
    		}
    		
    	}
    	
    	return busiestHour;
    }

    /**
     * Print the hourly counts.
     * These should have been set with a prior
     * call to analyzeHourlyData.
     */
    public void printHourlyCounts()
    {
        System.out.println("Hr: Count");
        for(int hour = 0; hour < hourCounts.length; hour++) {
            System.out.println(hour + ": " + hourCounts[hour]);
        }
    }
    
    /**
     * Print the lines of data read by the LogfileReader
     */
    public void printData()
    {
        reader.printData();
    }
}
\end{lstlisting}

\section*{Exercise 'factorial'}
The factorial algorithm based on iterations.
\begin{lstlisting}
public long factorial(int n)
{
	long result = 1;
	
	for (int i = 1; i <= n; i++)
	{
		if(i == 0)
		{
			result = 1;
		}
		else
		{
			result *=i;
		}
	}
	return result;
}
\end{lstlisting}

\section*{Exercise 'sum'}
The summation algorithm based on iterations.
\begin{lstlisting}
public long sum (int[] v, int first, int last)
{
	if(first == last)
		return v[first];
	else
	{
		long result = 0;
		
		for (int i = first; i <= last; i++)
		{
			result += v[i];
		}
		
		return result;
	}
}
\end{lstlisting}

\section*{Exercise 'max vector'}
Recursive and iterative ways of finding the maximum value of a vector. In order to use the recursive version, the \textit{v} variable should be set to \verb|vector.length-1| for this to work. Example implementation can be found in the end of this document.
\begin{lstlisting}
/**
 * Finds the maximum value in a vector recursively.
 * @param vector The vector.
 * @param v The length of the vector, i.e. vector.length-1.
 * @return The maximum value.
 */
public int maxValueInVectorRecursive(int[] vector, int v)
{
	if (v == 0)
	{
		return vector[0];
	}
	else
	{
		int temp = maxValueInVectorRecursive(vector, v-1);
		if(vector[v]  >  temp)
		{
			return vector[v];
		}
		else
		{
			return temp;
		}	
	}	
}

/**
 * Finds the maximum value in a vector using iteration.
 * @param vector The vector.
 * @return The maximum value.
 */
   public int maxValueInVector(int[] vector)
   {
   	int maxValue = Integer.MIN_VALUE;
   	
   	for(int value : vector)
   	{
   		if(value > maxValue)
   			maxValue = value;
   	}
   	
   	return maxValue;
   }
\end{lstlisting}

The example implementation. This a part of a test class used to validate this algorithm.
\begin{lstlisting}
public void	maxValueInVectorTest()
{
	int[] array = new int [50];
	for (int i = 0; i < 50; i++) 
	{
		array[i] = randInt(0, 20);
	}
	
	int max = maxValueInVector(array);
	
	System.out.println(max);
	
	int _max = maxValueInVectorRecursive(array, array.length-1);
	
	if(max != _max)
		fail("Max values not equal. Test 1");
	
	int[] vec = {3,2,1,5};
	
	if(maxValueInVector(vec) != maxValueInVectorRecursive(vec, vec.length -1))
		fail("Max value not equal. Test 2");
	
	System.out.println(_max);
}

// Thanks to @Greg Case at
// http://stackoverflow.com/questions/363681/
// generating-random-integers-in-a-range-with-java

public int randInt(int min, int max) {
    // NOTE: Usually this should be a field rather than a method
    // variable so that it is not re-seeded every call.
    Random rand = new Random();
    // nextInt is normally exclusive of the top value,
    // so add 1 to make it inclusive
    int randomNum = rand.nextInt((max - min) + 1) + min;
    return randomNum;
}
\end{lstlisting}


\end{document}

