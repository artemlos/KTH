\documentclass[10pt,a4paper]{article}
\usepackage[latin1]{inputenc}
\usepackage{amsmath}
\usepackage{amsfonts}
\usepackage{amssymb}
\usepackage{makeidx}
\usepackage{graphicx}
\usepackage[ruled,vlined]{algorithm2e}
\author{Artem Los}
\title{Hello}

\usepackage{marginnote}
\usepackage{verbatim} % for the box
\usepackage{fancyvrb} % for the box

\usepackage{listings}

\lstset{
% vilket spr�k vi anv�nder i v�ra kodlistings, s� att listings-paketet vet hur den ska highligta saker
language=Java,
% huruvida vi ska ha syntax highlighting
fancyvrb=true,
% hur stora tabstopp vi ska ha
tabsize=4,
% huruvida vi ska till�ta andra tecken �n a-z
extendedchars=\true
% hur breda listings vi vill ha (skriv exempelvis linewidth=0.5\textwidth f�r att f� listings som bara tar upp halva bredden av sidan)
linewidth=\textwidth,
% huruvida vi ska visa mellanslag
showstringspaces=false,
% huruvida vi ska bryta rader som �r f�r l�nga
breaklines=true,
% huruvida den ska f� bryta rader mitt i ord eller inte (true h�r betyder att den bara bryter mellan ord)
breakatwhitespace=true,
% indentera radbrytningar automatiskt
breakautoindent=true,
% l�gg in radnummer p� v�nster sida
numbers=left,
% hur stora radnumren ska vara
numberstyle=\tiny,
% hur l�ngt det ska vara mellan radnumren och koden
numbersep=8pt
}

\usepackage{pgf}
\usepackage{pgfpages}

\usepackage{fullpage}  % might require you to compile the page several times.



\begin{document}
\section*{Exercises 7.15-7.19}
\textit{Answer to 7.17}: Once a test fails, we can inspect the statement that caused it to fail. Eclipse will highlight that line of code. There appears to be a quick way (by right-clicking on the test case) to create a new task associated with the failure, so that other team members can look at it.
\subsection*{SalesItemTest}
\lstinputlisting{SalesItemTest.java}
\subsection*{CommentTest}
\lstinputlisting{CommentTest.java}

\section*{Exercise 'insertion sort'}
\subsection*{Sort class}
\lstinputlisting{Sort.java}

\subsection*{SortTest class}
\lstinputlisting{SortTest.java}

\section*{Exercise 'reverse order' of a vector}
The algorithm that is to be described below work as following: \textit{Given that an array contains $n$ elements, we take the first item and replace it with the last. Later on, we take the second item and replace it with the second last and so on. We continue this procedure until will reach $\lfloor n/2 \rfloor -1$}.
\begin{algorithm}
\SetKwData{Left}{left}\SetKwData{This}{this}\SetKwData{Up}{up}
\SetKwFunction{Union}{Union}\SetKwFunction{Swap}{Swap}
\SetKwInOut{Input}{input}\SetKwInOut{Output}{output}

\Input{An array $A$ of $n$ integers}
\Output{An array of integers in reveresed order}
\BlankLine
\For{$i \leftarrow 0$ \KwTo  $\lfloor n/2\rfloor -1$}{
\Swap($A[i]$, $A[n-i]$)}
\Return {A}
\caption{Reverse order of an integer array}
\end{algorithm}

\section*{Exercise 'order according to Big Oh'}

\begin{eqnarray*}
n+100 \\
n\log (n)\\
n^{1.5}\\
2^n\\
10^n
\end{eqnarray*}


\section*{Exercise 'Big Oh'}
Paul Bachmann's O-notation is defined as following:
\begin{equation}
f(n)=O(g(n)) \qquad \forall n
\end{equation}
which means that there is a constant $C$ such that:
\begin{equation}
|f(n)| \le C|g(n)| \qquad \forall n
\end{equation}
Big Omega notation is defined as
\begin{equation}
f(n) = \Omega(g(n)) \quad \iff \quad |f(n)| \ge C|g(n)| \quad \text{for some } C>0
\end{equation}

\begin{itemize}
\item $n(n+1)/2 = O(n^3)$ is true because $n(n+1)/2 = 0.5n^2 + 0.5n$ which grows slower than $n^3$. Using definition, we can always pick a constant $C$ to make it work.
\item  $n(n+1)/2 = O(n^2)$ is also true, similar to the reason above. When $n \to \infty$, only the highest degree term will matter. Again, we can always pick a constant to make this relationship valid.
\item $n(n+1)/2 = \Theta(n^3)$ isn't true because $n^3$ is not the lower bound, i.e. $n(n+1)/2 \not= \Omega(n^3)$
\item $n(n+1)/2 = \Omega(n)$ is true as $n$ grows slower than $0.5n^2+0.5n$
\end{itemize}

\section*{Exercise 'time complexity of algorithm'}
\begin{itemize}
\item The time complexity is $O(n^2)$
\item There is no 'good' case nor 'bad' case. The algorithm will have iterate the same number of times for a given size of $n$. Thus, $\Omega(f(n))$ of this algorithm is the same as the Big Oh.
\end{itemize}

\begin{algorithm}
\SetKwData{Left}{left}\SetKwData{This}{this}\SetKwData{Up}{up}
\SetKwFunction{Union}{Union}\SetKwFunction{Swap}{Swap}
\SetKwInOut{Input}{input}\SetKwInOut{Output}{output}

\Input{An array $A$ of $n$ integers}
\Output{An array $B$ of partial sums of $A$}
\BlankLine
\For{$i \leftarrow 1$ \KwTo  $n-1$}{
$B[i] = B[i-1] + A[i]$}
\Return {B}
\caption{Returns partial sums of a given array}
\end{algorithm}

This algorithm has time complexity $O(n)$.

\section*{Exercise 'function that is neither Oh nor Omega}
For example,

\begin{equation*}
f(x)=x^2 \times |sin(x)|
\end{equation*}

\end{document}

